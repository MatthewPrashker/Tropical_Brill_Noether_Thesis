\documentclass{article}

% Language setting
% Replace `english' with e.g. `spanish' to change the document language
\usepackage[english]{babel}

% Set page size and margins
% Replace `letterpaper' with`a4paper' for UK/EU standard size
\usepackage[letterpaper,top=2cm,bottom=2cm,left=3cm,right=3cm,marginparwidth=1.75cm]{geometry}

% Useful packages
\usepackage{amsmath}
\usepackage{graphicx}
\usepackage{tikz-cd}

\usepackage[colorlinks=true, allcolors=blue]{hyperref}

\title{Applications of Tropical Geometry to the Theory of Algebraic Curves}
\author{Matthew Prashker}


% Theorems, Lemmas, and Definitions Settings
\newtheorem{theorem}{Theorem}[section]
\newtheorem{corollary}{Corollary}[theorem]
\newtheorem{lemma}[theorem]{Lemma}
\newtheorem{definition}{Definition}[section]
\newtheorem{remark}{Remark}[section]


\begin{document}
\maketitle

\begin{abstract}
\noindent
We aim to give an expository account of a complete proof of the classical Brill-Noether theorem using techniques coming from Tropical Geometry.
\end{abstract}
\section{Introduction}
Given a smooth projective curve 
\section{Analogies between Curves and Graphs}
\label{analog}
\subsection{Definitions}
Our goal in this section is to develop analogs for classical notions from the theory of algebraic curves for so called $\textit{metric graphs}$. Metric graphs are essentially combinatorial objects which are interpreted topologically. First, by a weighted graph, we mean a connected multigraph without loops, in which each edge is assigned a positive real number. Associated to such a metric graph $G$, we get a compact connected topological space, denoted by $\Gamma (G)$, obtained by viewing edges of $G$ as line segments of distance given by the weight of that edge. We always view such weighted graphs up to equivalence, where two such graphs are equivalent if they admit a common length preserving sub-division. These metric graphs will play the role of algebraic curves for us, so they will be central objects in what follows. In fact, such a metric graph can naturally be viewed as a so called tropical curves as in [ref], although the latter is a more general type of object.  
\newline
\newline
Given a metric graph $\Gamma$, we want a notion of a divisor and a rational function on $\Gamma$. A divisor is simply defined to be an element of the free abelian group on the points of $\Gamma$, and an effective divisor is one with all non-negative coefficients. The definition of a rational function is slightly more complicated. As motivation for the definition, we look at rational functions on algebraic curves. Such rational functions can be locally expressed as the ratio of two polynomials. In the tropical setting, multiplication is replaced by addition, and addition is replaced by the $\textit{max}$ operation, so the natural analog for a rational function on a metric graph is a piecewise linear function (with finitely many pieces) with integer slopes. We denote the space of rational functions on a metric graph $\Gamma$ by $\mathcal{M}(\Gamma)$. Note that we can always replace $G$ by some sub-division to assume without loss of generality that $f$ is affine on the interior of each edge of $G$. In this case, we see that the associated divisor $(f)$ of some $f\in \mathcal{M}(\Gamma)$ is always supported on the vertices of $\Gamma$. Whenever we are considering a divisor on some metric graph, we always assume that we are working with a model of the metric graph in which the divisor is supported on the vertices of the graph. 
\newline
\newline
It is a classical fact that on a Riemann surface, the sum of the zeros and poles, counted with appropriate multiplicities, is zero. On a metric graph, we have the following
\begin{lemma}
Let $\Gamma$ be a metric graph and $f\in \mathcal{M}(\Gamma)$. Then $\text{deg}((f)) = 0$
\end{lemma}
$\textbf{Proof.}$ Take an equivalent subdivision of $G$ in which $f$ is affine on the interior of each edge. Then for every edge $e$ which is directed from $v_1$ to $v_2$ with the slope of $f$ on the interior of $e$ given by $n$, we see that $e$ contributes $n$ to $v_2$ and $-n$ to $v_1$ in the sum defining $\text{deg}((f))$. 
\begin{remark}
Whenever we consider a divisor on some metric graph $\Gamma$, we always assume that we are working with a model of $\Gamma$ with the property that the divisor is supported on the vertices of $\Gamma$. 
\end{remark}
Defining the notion of the rank of a divisor on a $\Gamma$ is slightly more complicated. Recall that in the case of an algebraic curve $X$ defined over some field $k$, the rank of a divisor $D$, denoted $r(D)$, is defined to be one less than the $k$ dimension of the vector space $H^{0}(X, \mathcal{O}_X(D))$, i.e. it is the dimension of the associated projective space $PH^{0}(X, \mathcal{O}_X(D))$. Equivalently, it is the dimension of the vector space of rational functions $f$ on $X$ such that $(f) - D$ is effective. If we try to make the analogous definition for a metric graph $\Gamma$, we run into the problem that the set of piecewise linear functions $\phi$ on $\Gamma$ with $(\phi) - D$ effective is not a vector space. (Give an example). Instead, we can use the following alternative definition of the rank of a divisor on $X$ to motivate the definition of the rank of a divisor on a metric graph. Given a divisor $D$, we can consider the associated line bundle $\mathcal{L}(D)$. Then $D$ has rank $r$ iff given any divisor $D^{\prime}$ of degree $r$, we can find some section $s\in H^{0}(X, \mathcal{L}(D))$ such that $(s) - D^{\prime}$ is effective. Concretely, in the case when all $r$ points are distinct, this means we can find some global section $s$ of $\mathcal{L}(D)$ which vanishes at these $r$ points. In other words, a divisor $D$ has rank at least $r$ iff for every divisor $D^{\prime}$ of degree $r$, the complete linear system $|D - D^{\prime}|$ is non-empty. 
\newline
\newline
In particular, every metric graph, as we have defined them, may naturally be viewed as a tropical curve. 
\newline
\newline
Now fix a metric graph $\Gamma$. We want to develop a notion of divisors and rational functions on this graph, in parallel to the theory of divisors and rational functions on a smooth projective curve. The analog of meromorphic functions on a smooth projective curve over the complex numbers (or more generally algebraic rational functions on a smooth projective curve over an arbitrary algebraically closed field) will be played by $\textit{piecewise-linear}$ functions on the metric graph $\Gamma$, which are defined as follows. Given such a piecewise linear function $\phi$ on $\Gamma$, we want to associate to $\phi$ a divisor $\text{div}(\phi)$ on $\Gamma$, in much the same way that we can associate a divisor to any non-zero rational function on a smooth project curve. Divisors of this form on $\Gamma$ will play the role of principal divisors. 
\section{Specialization Lemma}
Our goal in this section is to define Baker's specialization map and state and prove his Specialization Lemma from \cite{baker08}, which will be our main tool for relating geometric information about a curve to certain combinatorial information. First, by way of setup, throughout this section we let $R$ denote a complete discrete valuation ring,  with field of fractions $K$ and algebraically closed residue field $\kappa$. We also give ourselves a smooth, projective, geometrically connected curve $X$ over $K$, which is our main geometric object of interest. We want to fill in a diagram of the form
\[
\begin{tikzcd}
X \arrow[d] \arrow[r, hook]    & \mathcal{X} \arrow[d] & \mathcal{X}_{\kappa} \arrow[d] \arrow[l, hook'] \\
\text{Spec}(K) \arrow[r, hook] & \text{Spec}(R)        & \text{Spec}(\kappa) \arrow[l, hook']   \end{tikzcd}
\]
such that $\mathcal{X}$, which we think of as the total space of an infinitesimal family of curves over $\text{Spec}(R)$, is regular and flat over $\text{Spec}(R)$, and the special fiber $\mathcal{X}_{\kappa}$ is reduced with only ordinary double points as singularities and has smooth irreducible components $\{C_1, \cdots C_k\}$. Such a $\mathcal{X}$ is called a $\textit{strongly regular semistable model}$ of the curve $X$ (\textit{strong} refers to the smoothness of the irreducible components $C_i$). Both the regularity and the flatness of $\mathcal{X}$ over $\text{Spec}(R)$ will be used crucially in order to define the specialization map, as we will explain. 
\newline
\newline
It will be very useful in what follows to observe that Weil divisors on $\mathcal{X}$ (which are naturally in bijection with Cartier divisors as $\mathcal{X}$ is regular) fall into two categories. There are those which are supported on the special fiber $\mathcal{X}_{\kappa}$, which are commonly referred to as $\textit{vertical}$ divisors, and there are those which are Zariski closures of divisors supported on the generic fiber, referred to as $\textit{horizontal}$ divisors. Every divisor on $\mathcal{X}$ may be written uniquely as a sum of vertical and horizontal divisors.  
\newline
\newline
The combinatorial information referenced in the previous paragraph will be the so-called $\textit{dual graph}$ of the special fiber $\mathcal{X}_{\kappa}$. This graph, denoted $G(\mathcal{X}_{\kappa})$, has one vertex $v_i$ for each component $C_i$, and one edge between vertices $v_i$ and $v_j$ for each point of intersection of the components $C_i$ and $C_j$. Note that because each component $C_i$ is smooth by assumption, the graph $G(\mathcal{X}_{\kappa})$ has no loops, although it may have multiple edges, as per our conventions in [$\ref{analog}$]
\newline
\newline
We will need the following lemma on extending line bundles from the generic fiber in order to define Baker's Specialization Map:
\begin{lemma}
\label{extend}
Notation as above, any line bundle $L$ on the generic fiber $X$ extends to a line bundle $\mathcal{L}$ on the total space $\mathcal{X}$. Furthermore, such an extension is unique up to twisting by components of the special fiber, i.e. is of the form $\mathcal{L}\otimes_{\mathcal{O}_{\mathcal{X}}}\mathcal{O}_{\mathcal{X}}(C_i)$ after fixing one extension $\mathcal{L}$.  
\end{lemma}
\textbf{Proof.} To see the existence of an extension of $L$ to $\mathcal{X}$, represent $L$ by some Weil Divisor. Then the line bundle associated to the Zariski closure in $\mathcal{X}$ of this Weil divisor will give a desired extension. Because every Weil divisor on $\mathcal{X}$ can be written uniquely as a sum of horizontal and vertical divisors, we see that this extended Weil divisor is unique up to adding vertical divisors. But adding a vertical divisor exactly corresponds to twisting the associated line bundle by a component $C_i$ of the special fiber. 
\newline
\newline
We will now show how chip-firing enters the picture. The idea is that associated to each line bundle $L\in \text{Pic}(X)$, we will produce a divisor on the graph $G(\mathcal{X}_{\mathcal{\kappa}})$. This proceeds by first extending $L$ to $\mathcal{X}$, and then restricting $\mathcal{L}$ to each component $C_i$. Because each $C_i$ is a smooth curve, the restriction of $\mathcal{L}$ to $C_i$ has a well defined degree. This degree will be the coefficient of the vertex $v_i$ corresponding to the component $C_i$ in the divisor we are producing. We formalize this in the following definition.
\begin{definition}
Given a line bundle $\mathcal{L}\in \text{Pic}(\mathcal{X})$, the divisor $D(\mathcal{L})\in \text{Div}(G(\mathcal{X}_{\mathcal{\kappa}}))$ is defined by
\[
D(\mathcal{L}) := \sum_{i = 1}^{k} \text{deg}(\mathcal{L}|_{C_i})v_i,
\]
where $v_i$ is the vertex corresponding to the component $C_i$. 
\end{definition}
If we want to extend this definition to map line bundles on $X$ to divisors on $G(\mathcal{X}_{\mathcal{\kappa}})$, the issue we run into is that the extension $\mathcal{L}$ in not unique. However, by Lemma \ref{extend}, the extension is unique up to twisting by a component of the special fiber. Thus, in order to understand the failure of this non-uniqueness, we just need to understand how $D(\mathcal{L})$ changes when we twist $\mathcal{L}$ by some $C_i$. For this, we have the following lemma, which brings the combinatorics of chip-firing into the picture.
\begin{lemma}
Given a line bundle $\mathcal{L}\in \text{Pic}(\mathcal{X})$ and a component $C_i$ of $\mathcal{X}_{\kappa}$, the divisor
\[
\text{Div}(\mathcal{L}\otimes_{\mathcal{O}_{\mathcal{X}}}\mathcal{O}_{\mathcal{X}}(C_i))
\]
is obtained from the divisor $\text{Div}(\mathcal{L})$ by performing a chip-firing move the vertex $v_i$. 
\end{lemma}
\textbf{Proof} It will be convenient for the proof to work in terms of Weil divisors on $\mathcal{X}$ instead of line bundles, which is allowed as $\mathcal{X}$ is regular. 
\newline
\newline
We will also need the following lemma on comparing the degrees of the restrictions of line bundles on $\mathcal{X}$ to the generic and special fibers:
\begin{lemma}
Notation as above, let $\mathcal{L}$ be a line bundle on $\mathcal{X}$. Then we have the equality
\[
\text{deg}(\mathcal{L}|_{X}) = \text{deg}(\mathcal{L}_{\mathcal{X}|_{\kappa}})
\]
\end{lemma}
\textbf{Proof.} Using Riemann-Roch on the smooth curve $X$, we have 
\[
\text{deg}(\mathcal{L}|_X)= \mathcal{X}(\mathcal{L}_X) + g - 1
\]
Because $\mathcal{X}$ is flat over $\text{Spec}(R)$, using the fact that the Euler characteristic of line bundle is constant for flat families, we have 
\[
\mathcal{X}(\mathcal{L}|_X) = \mathcal{X}(\mathcal{L}_{\mathcal{X}_{\mathcal{\kappa}}})
\]
By considering the pullback of $\mathcal{L}|_{\mathcal{X}_{\kappa}}$ to the normalization of the special fiber, we conclude that 
\[
\mathcal{X}(\mathcal{L}_{\mathcal{X}_{\mathcal{\kappa}}}) = \sum_{i = 1}^{k}\mathcal{X}(\mathcal{L}_{C_i}) = k - 
\sum_{i = 1}^{k}g(C_i)
\]
\newline
\newline
\newline
\newline
The source of Baker's Specialization map will be the group $\text{Pic}(X)$ of line bundles on $X$. To describe the target, we need the following definition which associated a group to any graph (recall our conventions on graphs from \ref{analog}). 
\bibliographystyle{alpha}
\bibliography{references}

\end{document}